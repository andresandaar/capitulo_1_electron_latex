%link de acceso directo


\hypertarget{linkapp}{APP: Google Play.}

\parbox {15cm} { \raggedright URL1: \url{https://play.google.com/store/apps/details?id=com.teacapps.barcodescanner} }


\hypertarget{CURSOELECTRON}{Canal de YouTube: John Ortiz Ordoñez.Curso de Electron}

\parbox {15cm} { \raggedright URL2: \url{https://youtube.com/playlist?list=PL2PZw96yQChzi-lLw6rqqAPRkNXSOeqtr}}

\hypertarget{NODE2}{Canal de YouTube: Códigofacilito.Ejemplo Node.js}

URL3: \url{https://youtu.be/wd8zf3D0jic}

\hypertarget{code}{Página Oficial: Visual Studio Code.}

URL4: \url{https://code.visualstudio.com/}

\hypertarget{node1}{Página Oficial: Node.js.}

URL5: \url{https://nodejs.org/es/}

\hypertarget{oficialpagina}{Página Oficial: Electron.js.}

URL6: \url{https://www.electronjs.org/}

\hypertarget{tutorial}{Tutorial Educativo Electron.}

URL7: \url{https://www.electronjs.org/docs/latest/tutorial/quick-start}

\hypertarget{git}{Página Oficial de Git}

URL8: \url{https://git-scm.com/}

\hypertarget{videogit}{Videotutorial Intalacion de Git}

URL9: \url{https://youtu.be/h9ZH2wFpSUc}

\hypertarget{texgit}{Tutorial Web Intalacion de Git}

\parbox {15cm} { \raggedright URL10: \url{https://www.mclibre.org/consultar/informatica/lecciones/git-instalacion.html} }

\hypertarget{github}{Página Oficial de GitHub}

URL11: \url{https://github.com/}

\hypertarget{tutogithubt}{Tutorial web de registro GitHub}

\parbox {15cm} { \raggedright URL12: \url{https://git-scm.com/book/es/v2/GitHub-Creaci \% C3 \% B3n-y-configuraci \%C3 \% B3n-de-la-cuenta} }






\newpage
%@reference{linkapp,
%author = "Gogle play",
%url = "https://play.google.com/store/apps/details?id=com.teacapps.barcodescanner",
%}
%@reference{node,
%author = "codigofacilito",
%url = "https://youtu.be/wd8zf3D0jic",
%}

%@reference{electronjs,
%author = "John Ortiz Ordoñez",
%url = "https://youtube.com/playlist?list=PL2PZw96yQChzi-lLw6rqqAPRkNXSOeqtr",
%}
%@reference{code,
%author = "visual code",
%url = "https://code.visualstudio.com/",
%}
%@reference{node1,
%author = "node js pagina",
%url = "https://nodejs.org/es/",
%}
%@reference{oficialpagina,
%author = "electron pagina",
%url = "https://www.electronjs.org/",
%}
%@reference{tutorial,
%author = "electron pagina",
%url = "https://www.electronjs.org/docs/latest/tutorial/quick-start",
%}
