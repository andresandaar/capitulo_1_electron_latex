
Electron es un framework para crear aplicaciones de escritorio usando \textbf{JavaScript, HTML y CSS}. Incrustando \textbf{Chromium y Node.js} dentro del mismo, Electron le permite mantener una base de código JavaScript y crear aplicaciones multiplataforma que funcionan en \textbf{ Windows, macOS y Linux}, no requiere experiencia en desarrollo nativo.
Miles de organizaciones que abarcan todas las industrias utilizan Electron para construir Software multiplataforma.\cite{elec}

Alguna de las aplicaciones más famosa creadas por electron son:

\begin{tabular}{l l}
    \textcolor{red}{ $\bigstar $} Visual Studio Code & \begin{minipage}{1.5cm}
     \includegraphics[width=0.4\textwidth]{img/code.jpg}
     \end{minipage}\\
     &\\
     \textcolor{red}{$\bigstar $} Facebook Messenger &\begin{minipage}{1.5cm}
      \includegraphics[width=0.4\textwidth]{img/faceme.jpg}
     \end{minipage}\\
     &\\
     \textcolor{red}{$\bigstar $} Twitch &\begin{minipage}{1.5cm}
     \includegraphics[width=0.4\textwidth]{img/twitch.png}
     \end{minipage}
\end{tabular}

\newpage

\subsection{¿QUÉ ES CHROMIUM?}

Es una base de código abierto para desarrollar un navegador web, mantenida por diversas compañías que posteriormente usan el código fuente para crear su propia versión de navegador con características adicionales.
Además, es multiplataforma, ya que funciona con Windows, Linux, Mac y Android. Asimismo, cuenta con soporte para las extensiones de Chrome Store.\cite{Chromium}

\subsection{¿QUÉ ES NODE.JS?}

Node.js, es un entorno en tiempo de ejecución multiplataforma para la capa del servidor (en el lado del servidor) basado en JavaScript.
Este entorno es controlado por eventos diseñados para crear aplicaciones escalables, permitiéndote establecer y gestionar múltiples conexiones al mismo tiempo. Gracias a esta característica, no tienes que preocuparte con el bloqueo de procesos, pues no hay bloqueos.\cite{Nodejs}

\subsubsection{¿CÓMO FUNCIONA NODE.JS?}
El diseño de Node.js está inspirado en sistemas como el Event Machine de Ruby o el Twisted de Python. Sin embargo, Node.js presenta un bucle de eventos como una construcción en tiempo de ejecución en lugar de una biblioteca. Este bucle de eventos es invisible para el usuario.

Otra característica especial de Node.js es que está diseñado para simplificar la comunicación. No tiene subprocesos, pero te permite aprovechar múltiples núcleos en su entorno y compartir sockets entre procesos.\cite{Nodejs}


\subsubsection{¿PARA QUÉ SIRVE NODE.JS?}
Puedes utilizar Node.js para diferentes tipos de aplicaciones. Los siguientes son algunos de los ejemplos:

\begin{itemize}
   \renewcommand{\labelitemi} {\textcolor{red}{$\bigstar$}}
    \item Aplicaciones de transmisión de datos (streaming)
    \item Aplicaciones intensivas de datos en tiempo real
    \item Aplicaciones vinculadas a E/S
    \item Aplicaciones basadas en JSON:API
    \item Aplicaciones de página única\cite{Nodejs}
\end{itemize}
\newpage
Node.js también se integra dentro de la gestión de paquetes utilizando la herramienta NPM (Node Package Manager; esta se instala por defecto con Node.js), la cual nos permite instalar numerosos componentes a través de un repositorio en línea.
Ejemplo del uso de Node.js \hyperlink{EJEMPLONODE}{\textcolor{red}{Clic aquí para ver} \textcolor{red}{$\looparrowleft$}}

\subsubsection{¿QUÉ ES Y PARA QUÉ SIRVE NPM?}

\begin{figure}[H]
    \centering
    \includegraphics[width=0.6\textwidth]{img/NPM.png}
\end{figure}

De sus siglas \hypertarget{camannpm}{\textbf{NPM (Node Package Manager) }}es un gestor de paquetes desarrollado en su totalidad bajo el lenguaje JavaScript por Isaac Schlueter, a través del cual podemos obtener cualquier librería con tan solo una sencilla línea de código, lo cual nos permitirá agregar dependencias de forma simple, distribuir paquetes y administrar eficazmente tanto los módulos como el proyecto a desarrollar en general.\cite{Npmjs}





