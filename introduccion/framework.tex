
El Framework es una especie de plantilla, un esquema conceptual, que simplifica la elaboración de una tarea, ya que solo es necesario complementarlo de acuerdo a lo que se quiere realizar. \cite{Frame}

\subsection{USO DE LOS FRAMEWORK EN INTERNET}

Para cualquier proyecto en Internet se requiere un desarrollador web que produzca el software o la aplicación que necesitamos.
Dependiendo del tipo de proyecto, esta tarea puede durar mucho tiempo si se crea de la nada. Es necesario elaborar parte por parte, haciendo pruebas y aciertos hasta conseguir el objetivo.
Todo esto puede requerir uno o más programadores, además del tiempo suficiente para realizar las pruebas necesarias hasta que el software esté funcionando perfectamente.
Sin embargo, los Framework permiten entregar un proyecto en menos tiempo y con un código más limpio, cuya eficacia ya ha sido comprobada.
A partir del Framework los programadores pueden complementar y/o modificar la estructura base para entregar el software o la aplicación que cumpla los objetivos requeridos.\cite{Frame}